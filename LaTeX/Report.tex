\chapter{}
\section{Вступні відомості}
\noindent\textbf{Мета роботи:} Ознайомлення з принципами баєсівського підходу в криптоаналізі, побудова детерміністичної та стохастичної 
вирішуючих функцій для моделей схем шифрування та криптоаналіз моделей шифрів за допомогою програмної реалізації, зокрема здійснення 
порвіняльного аналізу вирішуючих функцій.

\noindent\textbf{Постановка задачі:}
\begin{enumerate}
    \item Створіть репозиторій у системі контролю версій Git/GitHub;
    \item Реалізуйте алгоритми програмно та представите результати побудови детермінованих та стохастичних вирішальних функцій у 
        вигляді таблиць. Для цього необхідно:
        \begin{enumerate}
            \item обчислити розподіли $P(C)$ та $P(M, C)$;
            \item на основі цих розподілів обчислити $P(M \vert C)$;
            \item побудова оптимальних детермінованих та стохастичних вирішальних функцій зводиться до максимізації $P(M \vert C)$.
        \end{enumerate}
    \item Розрахуйте середні втрати, проведіть порівняльний аналіз функцій прийняття рішень.
    \item Підготувати звіт для комп'ютерного практикуму.
\end{enumerate}
\section{Результати виконання роботи. Варіант 15}
\begin{figure}[!ht]
    \centering
    \begin{minipage}{0.95\linewidth}
        \includegraphics[width=0.9\textwidth, scale = 2.0]{ReportPic/report_1.png}
    \end{minipage}
\end{figure}
\newpage
\begin{figure}[!ht]
        \centering
        \begin{minipage}{0.95\linewidth}
            \includegraphics[width=0.95\textwidth, scale=1.2]{ReportPic/report_2.png} % $P (C_{k}) = \sum\limits_{i, j : (i, j) = k} P(M_{i}) \cdot P(K_{j})$
        \end{minipage}
\end{figure}
\begin{figure}[!ht]
        \centering
        \begin{minipage}{0.97\linewidth}
            \includegraphics[width=0.97\textwidth, scale=1.5]{ReportPic/report_3.png}
        \end{minipage}
\end{figure}

\section{Побудова вирішуючих функцій}
\begin{definition}    
    ~\par Оптимальна (баєсівська) детерміністична функція [в межах лабораторної роботи] визначається наступним чином: 
    \begin{equation*}
        \delta_{B} = \left\{\delta_{B}^{(n)} : \mathcal{M} \rightarrow \mathcal{C}\right\}, 
    \end{equation*}
    де $P \left(\delta_{B}^{(optim)} \vert C\right) = \max\limits_{m \in M} P \left(M_{i} \vert C\right)$. \\ 
    Тобто фактично детерміністична функція дорівнює довільному шифротексту, який дорівнює максимальному значенню в 
    $i$-тому рядку таблиці.
\end{definition}
\begin{definition}
    ~\par Стохастична розв'язувальна функція $\delta_{D}$ є оптимальною тоді і тільки тоді, коли $\forall \, n$ 
    з нерівності $\delta_{c}^{(n)} \left(C, M\right) > 0$ випливає, що $P \left(M \vert C\right) = \max\limits_{M'} P \left(M' \vert C\right)$.
    Тобто 
    \begin{equation*}
        \delta_{D}^{optim} \left(C, m\right) = 
        \begin{cases}
            \frac{1}{G}, \text{ if } P \left(M \vert C\right) = \max\limits_{M'} P \left(M' \vert C\right) \\
            0, \text{ otherwise}
        \end{cases},
    \end{equation*}
    де $G$ -- максимальна кількість відкритих текстів $M$, які мають найбільшу [однакову] ймовірність для обраного шифротексту $C$.
\end{definition}
\newpage

\begin{figure}[!ht]
    \centering
    \begin{minipage}{0.9\linewidth}
        \includegraphics[width=0.9\textwidth, scale=1.2]{ReportPic/report_4.png}
    \end{minipage}
\end{figure}
\begin{figure}[!ht]
        \centering
        \begin{minipage}{0.9\linewidth}
            \includegraphics[width=0.9\textwidth, scale=1.0]{ReportPic/report_5.png}
        \end{minipage}
\end{figure}

\section{Висновки:}
Подивившись на отримані результати середніх втрат можна впасти в ступор, оскільки вони виявилися однаковими. На нашу думку це 
може бути бути пов'язано з недостатньою точністю обрахунків. Маємо припущення, що стохастична (a.k.a. випадкова) вирішуюча 
функція мала б відповідати більшій кількості потенційних ВТ до відповідно обраного ШТ, порівняно зі строго детерміністичною. 
Вона також могла показувати як зашкально добрий результат, так і навпаки (жартуємо, будь-яку випадковість можна передбачити). 
Варто зазначити, що при збільшенні кількості вхідних даних, стохастична вирішуюча функція (яка являє собою багаторозмірну 
матрицю) буде займати багатенько пам'яті, що може сповільнити процес виконання програми.

% \begin{tblr}{
%     colspec={|c|*{20}{c|}},
%     row{1} = {font=\bfseries}
%     }
%     Row & C0       & C1        & C2        & C3        & C4        & C5       & C6        & C7        & C8        & C9        & C10      & C12       & C12       & C13       & C14       & C15       & C16       & C17       & C18       & C19       \\
%     0   & 0        & 0.102804  & 0.092437  & 0         & 0         & 0.111111 & 0.0894309 & 0         & 0.0728477 & 0.0956522 & 0        & 0.0814815 & 0         & 0.0748299 & 0         & 0.0791367 & 0         & 0.0541872 & 0.0956522 & 0.0894309 \\
%     1   & 0        & 0         & 0         & 0         & 0.0814815 & 0        & 0.0894309 & 0.0956522 & 0.0728477 & 0.0956522 & 0        & 0.0814815 & 0         & 0         & 0.346847  & 0.0791367 & 0.0866142 & 0.0541872 & 0.0956522 & 0.0894309 \\
%     2   & 0        & 0         & 0         & 0         & 0.0814815 & 0        & 0.0894309 & 0.0956522 & 0         & 0         & 0.111111 & 0         & 0.0866142 & 0.0748299 & 0         & 0.0791367 & 0         & 0.0541872 & 0.0956522 & 0.0894309 \\
%     3   & 0.120879 & 0.102804  & 0.092437  & 0         & 0.0814815 & 0        & 0         & 0.0956522 & 0.0728477 & 0.0956522 & 0        & 0.0814815 & 0.0866142 & 0.0748299 & 0         & 0         & 0.0541872 & 0         & 0                     \\
%     4   & 0.120879 & 0.102804  & 0         & 0.092437  & 0         & 0        & 0         & 0.0956522 & 0.0728477 & 0.0956522 & 0.111111 & 0         & 0         & 0         & 0.0990991 & 0.0791367 & 0.30315   & 0.0541872 & 0.0956522 & 0.0894309 \\
%     5   & 0        & 0.0280374 & 0.0252101 & 0.0252101 & 0         & 0        & 0.0243902 & 0.026087  & 0         & 0.026087  & 0.030303 & 0.0222222 & 0         & 0.0204082 & 0         & 0.0755396 & 0.023622  & 0.0147783 & 0.026087  & 0         \\
%     6   & 0        & 0.0280374 & 0.0252101 & 0.0252101 & 0.0222222 & 0.030303 & 0         & 0         & 0.0198675 & 0         & 0        & 0.0222222 & 0.023622  & 0.0204082 & 0.027027  & 0         & 0.023622  & 0         & 0         & 0.0243902 \\
%     7   & 0.032967 & 0         & 0.0252101 & 0         & 0         & 0.106061 & 0.0243902 & 0         & 0.0198675 & 0         & 0.030303 & 0         & 0         & 0.0204082 & 0         & 0.0215827 & 0.023622  & 0.0147783 & 0         & 0         \\
%     8   & 0.115385 & 0.0280374 & 0         & 0.0252101 & 0         & 0.030303 & 0         & 0         & 0         & 0.026087  & 0.030303 & 0.0222222 & 0.023622  & 0         & 0         & 0.0215827 & 0.023622  & 0         & 0.026087  & 0.0243902 \\
%     9   & 0.032967 & 0         & 0         & 0.0252101 & 0.0222222 & 0.030303 & 0.0853659 & 0.026087  & 0.0198675 & 0         & 0.030303 & 0.0222222 & 0         & 0.0204082 & 0.027027  & 0         & 0.023622  & 0         & 0.026087  & 0.0243902 \\
%     10  & 0.032967 & 0.0280374 & 0.0252101 & 0         & 0.0222222 & 0.030303 & 0.0243902 & 0         & 0.0198675 & 0.026087  & 0        & 0         & 0.023622  & 0         & 0.027027  & 0.0215827 & 0         & 0.0147783 & 0.026087  & 0         \\
%     11  & 0        & 0.0280374 & 0         & 0.0252101 & 0.0222222 & 0.030303 & 0.0243902 & 0.026087  & 0.0198675 & 0.026087  & 0        & 0         & 0         & 0         & 0.027027  & 0         & 0.0826772 & 0.0147783 & 0.026087  & 0.0243902 \\
%     12  & 0.032967 & 0.0981308 & 0.0252101 & 0.0252101 & 0.0222222 & 0.030303 & 0         & 0         & 0         & 0.026087  & 0        & 0.0222222 & 0         & 0         & 0.027027  & 0.0215827 & 0         & 0         & 0.026087  & 0.0243902 \\
%     13  & 0.032967 & 0.0280374 & 0.0882353 & 0         & 0.0222222 & 0.030303 & 0.0243902 & 0.026087  & 0         & 0         & 0        & 0.023622  & 0         & 0.027027  & 0.0215827 & 0.023622  & 0         & 0.026087  & 0.0243902             \\
%     14  & 0        & 0.0280374 & 0.0252101 & 0.0252101 & 0.0222222 & 0        & 0.0243902 & 0.026087  & 0.0198675 & 0.026087  & 0.030303 & 0.0222222 & 0.023622  & 0.0714286 & 0         & 0         & 0         & 0.026087  & 0                     \\
%     15  & 0.032967 & 0.0280374 & 0.0252101 & 0         & 0         & 0        & 0         & 0.026087  & 0         & 0.026087  & 0        & 0         & 0.0204082 & 0.027027  & 0         & 0.023622  & 0         & 0.026087  & 0.0243902             \\
%     16  & 0        & 0.0280374 & 0.0252101 & 0.0252101 & 0         & 0.030303 & 0.0243902 & 0.026087  & 0.0198675 & 0.026087  & 0.030303 & 0         & 0.023622  & 0         & 0.027027  & 0.0215827 & 0         & 0         & 0.0913043 & 0         \\
%     17  & 0.032967 & 0         & 0         & 0.0252101 & 0         & 0.030303 & 0.0243902 & 0         & 0         & 0         & 0.030303 & 0.0222222 & 0         & 0.0204082 & 0.027027  & 0.0215827 & 0.023622  & 0         & 0.0913043 & 0.0243902 \\
%     18  & 0.032967 & 0.0280374 & 0         & 0         & 0         & 0        & 0         & 0.026087  & 0.0198675 & 0         & 0.106061 & 0.0222222 & 0.023622  & 0.0714286 & 0.027027  & 0.0215827 & 0.023622  & 0.0147783 & 0         & 0         \\
%     19  & 0.032967 & 0.0280374 & 0         & 0.0252101 & 0.0222222 & 0.030303 & 0.0243902 & 0.026087  & 0.0695364 & 0.026087  & 0.030303 & 0.0222222 & 0         & 0.0204082 & 0         & 0.0215827 & 0         & 0         & 0         & 0         \\
% \end{tblr}